\newpage
\newcommand{\tituloConsideracoesFinais}{Considerações finais}

\chapter*{\tituloConsideracoesFinais}
\addcontentsline{toc}{section}{\MakeUppercase{\tituloConsideracoesFinais}}

Chegar às considerações finais é, por si só, uma conquista digna de nota — quiçá de um breve testemunho na próxima aula de espiritualidade. Ao longo desta jornada entre citações, glosas e normas de formatação, procurou-se demonstrar que a tarefa de elaborar uma monografia teológica é tão santificadora quanto participar de uma reunião ordinária do presbitério com pauta extensa e café fraco.

O texto que aqui se apresenta não tem a pretensão de revolucionar o ensino teológico, tampouco substituir os bons e velhos sermões expositivos de 45 minutos por modelos .tex. Sua função é simples: servir como guia, consolo e, em certos momentos, catarse para aqueles que trilham o caminho do bacharelado em Teologia com mais dúvidas que certezas, mais marcações no texto que nas Escrituras, e mais revisões que horas de sono.

Ressalte-se que a tradição reformada, ancorada na \gls{cfw}, valoriza não apenas a ortodoxia, mas também a ordem — inclusive a ordem das seções da monografia. Assim, este trabalho cumpre duplamente seu papel: honra o Deus da verdade e agrada aos reverendíssimos avaliadores, ou ao menos não lhes oferece argumentos demasiadamente fáceis para reprovação sumária.

Por fim, se este modelo inspirar ao menos um seminarista a terminar seu \gls{tcc} com mais alegria do que angústia, já terá cumprido seu propósito. Se, além disso, conseguir fazê-lo rir entre um parágrafo e outro, então terá alcançado algo ainda mais raro nos corredores do seminário: graça com leveza.
